\chapter[Overview]{Overview of the Project}

\section{Introduction}

The evolution of software development has increased the demand for efficient design tools. UML diagrams play a vital role by bridging conceptual design and implementation, improving stakeholder communication and offering standardized documentation.

Traditional approaches to diagramming face challenges due to manual effort and technical complexity. With the rise of AI and LLM technologies, automating diagram generation is now feasible.

This project proposes a platform combining natural language accessibility with the precision of textual UML generation, democratizing the process while maintaining professional standards.

\section{Presentation of the Project Context}

\subsection{Problem Statement}

UML creation using GUI and textual tools presents several challenges:

\textbf{GUI-Based Tools:} Difficult to master, time-consuming, limited collaboration, and weak version control integration.

\textbf{Textual Tools:} Require syntax knowledge (e.g., PlantUML, Mermaid), lack real-time feedback, and pose debugging difficulties.

\textbf{Integration Issues:} Poor workflow integration and limited automation.

\subsection{Existing Solutions}

\textbf{AI-Based Tools:} Use LLM to generate diagrams from user input, but often lack accuracy.Tools like ChatUML\cite{1} and DiagrammingAi\cite{2} provide fast feedback but lack support for complex cases.

\textbf{Limitations:} Existing tools lack full AI integration, offer inconsistent quality, and miss collaborative/community features.

\subsection{Proposed Solution}

\textbf{Core Idea:} The platform combines LLM with PlantUML to interpret natural language and generate accurate diagrams.

\textbf{Key Features:} Real-time validation, collaborative editing, version control support, and a marketplace for sharing templates.

\textbf{Architecture:} Microservices separate LLM, generation, and UI layers for scalability.

\textbf{Advantages:} Professional-quality output, community-oriented design, and user-friendly interfaces.


\section{Methodology Agile and Scrum Framework}

Agile promotes iterative development and adaptability, ideal for evolving AI projects. Scrum enhances Agile through defined roles (Product Owner, Scrum Master, Development Team), events (Planning, Daily, Review, Retrospective), and artifacts (Product Backlog, Sprint Backlog, Increment).

This framework ensures regular inspection, collaboration, and adaptation, supporting continuous improvement throughout the development process.


\section{Conclusion}
This project addresses key limitations in UML generation by integrating LLM with PlantUML.
The proposed platform enhances usability, collaboration, and automation through a scalable architecture.
By leveraging AI, it democratizes diagramming while maintaining professional quality and precision.
